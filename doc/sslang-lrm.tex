\documentclass{article}
\usepackage{times}

\usepackage{listings}
\usepackage{xcolor}
\usepackage{hyperref}
\usepackage{syntax}

\definecolor{listingbackground}{rgb}{0.8,0.8,0.8}

\title{SSLANG Language Reference Manual}
\author{Hans Montero, John Hui, and Stephen A. Edwards}
\date{2021}

\lstdefinelanguage{sslang}{
  morekeywords={let,if,then,else,while,after,wait,pipe},
  morecomment=[l]{//},
  morecomment=[s]{/*}{*/},
  columns=flexible,
}

\lstset{language=sslang,
  columns=flexible,
  commentstyle={\itshape\color{red}},
  backgroundcolor=\color{listingbackground},
}

\begin{document}

\maketitle

\section{Wish List}

\subsection{Philosphy}

\textit{SE:} Make the IR, especially, as simple as possible (a la
GHC). Take a Reynolds-like approach where the larger language has more
features that boil down into simpler primitives.  E.g., typeclasses
can be expressed as dictionaries with function pointers and completely
evaporate.  I'm inspired by the make-it-explicit approach to garbage
collection operations in the Perceus tech report (Reinking et al.,
MSR-TR-2020-42).

\subsection{Syntax}

\textit{SE:} Minimal punctuation (e.g., print ``hello'' vs. print(``hello'')).
\textit{JH:} I would like to have punctuation for types/operations with special
built-in meaning, e.g., \verb|&| or \verb|*| for references. I'm not opposed to
using parens for function application, but it depends on whether we can have
partial application.

\textit{SE, JH:} Nested comments

\textit{JH:} Single-line comment syntax. I think this makes sense especially for
a whitespace-sensitive language.

\textit{SE, JH:} Indentation-sensitive nesting; keywords start blocks (Haskell-style).

\textit{SE, JH:} User-defined binary operators and precedence levels.

\textit{JH:} User-defined type-level operator definitions. e.g., \verb|$| for
avoiding parens in type application, \verb|&| for ref.

\textit{JH:} some kind of readable, forward-thinking syntax for annotations that
the compiler can take advantage of, e.g., must-inline, dont-inline,
stack-allocate, maybe-unused. Should be less hacky than Haskell's
\verb|{-# LANGUAGE ... #-}| and C/C++'s \verb|#pragma ..| and
\verb|__attribute__((..))|.

\textit{SE:} C\# has an elaborate attributes mechanism.
Most declarations allow attributes to be placed in square brackets before.  \url{https://docs.microsoft.com/en-us/dotnet/csharp/programming-guide/concepts/attributes/}

\begin{lstlisting}{language=[Sharp]C}
  
[Serializable]
public class SampleClass
{
    // Objects of this type can be serialized.
} 
\end{lstlisting}

\subsubsection{Discussions}

\textit{SE:} What should function definitions (type signatures) look like?
Haskell-style or OCaml-style?  Look at the Nim language.  Named arguments?  Make
it look like data constructors?
\textit{JH:} Should be OCaml-style, in that it is inline, but it should not look
so ugly. Whatever syntax we use, declaration for follow invocation.

\begin{itemize}
  \item OCaml uses something like: \verb|f (a: int) (b: int) = ..|. Too many
    parentheses.

  \item In our meeting, we threw around the idea of something like:
    \verb|f a: Int, b: Int = ..|. \textit{JH:} seems fine, but looks weird because comma
    precedence is not widely understood.

  \item Rust, Go, Javascript, and other recent languages use: \verb|f(a: Int, b: Int) .. |.
    \textit{JH:} I don't love it, but I don't hate it either, and it's familiar.

  \item SML allows tuple syntax for function arguments:
    \verb|fun mult x y = x * y| has type
    \verb|int -> int -> int|,
    but \verb|fun mult (x, y) = x * y| has type
    \verb|int * int -> int|.

\begin{verbatim}

foo () -> Int = ..       -- Void function
----
f a: Int -> Int = ..   -- Interpret as '(Int -> Int) is the type of f'
f (a: Int) -> Int = .. -- Interpret as 'a is of type Int and f returns Int'
----

-- g: (Int, Int) -> Int

g (a: Int, b: Int) -> Int = .. -- Looks Pythonic, but g actually takes 1 tuple arg

g ((a, b): (Int, Int)) -> Int = ... -- OCaml-like

g (a, b): (Int, Int) -> Int = ... -- Interpret as '(Int, Int) -> Int is the type of g'

----

-- h: Int -> Int -> Int

h a b: Int -> Int -> Int = ... -- 'Int -> Int -> Int is the type of h'

h: Int -> Int -> Int = \a = \b = ...  -- OK, if single ID, then expect colon, like global var

h (a: Int) (b: Int) -> Int = ... -- OK, should parse, OCaml-like 

----
Top Level Definition Grammar

  TOPDEF ::= ID() -> TYPE = { BODY }        -- Void function
           | ID PARAM_T+ -> TYPE = { BODY } -- 'Inline' type signature
           | ID PARAM_T* : TYPE = { BODY }  -- 'Postfix functional' type signature

  PARAM_T ::= ID               -- Haskell-style, only param name
            | (PARAM)          -- OCaml-style, param wrapped in parens for typing
            | (PARAM(,PARAM)+) -- Tuple, think Python

  PARAM ::= (PARAM)          -- Unwrap parens
          | ID(: TYPE)?                -- Optionally typed identifier
          | (PARAM(,PARAM)+) (: TYPE)? -- Optionally typed tuple

  TYPE ::= [TYPE]         -- Desugars into List TYPE
         | TYPE -> TYPE
         | (TYPE)
   | (TYPE(,TYPE)+) -- Desugars into (,,) t1 t2 t3 
         | Int8 | Int16 | Int32 | Int64
         | UInt8 | UInt16 | UInt32 | UInt64
         | Bool
         | Event
   | &TYPE          -- TODO: Desugars into Ref TYPE
         | Sched TYPE

---

data Type = TypeID String
    | TApply Type Type
    | ...

TODO for Stephen:
- flesh out the lexer documentation
\end{verbatim}

\end{itemize}

Syntax for type expressions: in addition to nullary types, we want to have
higher-arity type operators, i.e., \verb|( , )|, \verb|->|, \verb|&|, etc.
Broadly speaking, the things on the right of a colon.

\subsection{Types and expressions}

\textit{SE, JH:} Type expressions.

\textit{SE, JH:} Algebraic data types + pattern-matching.

Statically sized arrays passable through abstract array arguments, e.g., so you
can write a sort that's polymorphic in the size of the array; implies array size
is a runtime value, not (just) compile-time.

\textit{SE, JH:} Typeclasses at least for arithmetic operators

Pass-by-value and pass-by-reference values

\textit{JH:} References to arrays of T, arrays of arrays of T

\textit{JH, SE:} Array slices (``safe'' references to parts of arrays that play
well with parallelism)

\textit{SE, JH:} Type inference at least for local variables

\textit{JH:} Type annotations for function arguments

\textit{SE, JH:} Parametric polymorphism, likely eliminated by whole-program
monomorphisation (type variables)

\textit{SE, JH:} A notion of constants (assign once) versus variables (assign
multiple times)

\textit{JH:} a notion of ownership that users can opt into (which reduces to
reference-sharing otherwise).

\subsubsection{Discussions}

Partial application?  Or do we prohibit it?

\subsection{Control flow}

\textit{SE, JH:} Parallel function calls.

\textit{SE, JH:} Wait on variables

\textit{SE, JH:} Immediate and delayed assignment ``primitives''.

\textit{JH:} Some sort of exception handling, which supports missed deadlines.

\textit{SE, JH:} combinators to derive "wait until" and "wait both" and "wait either".

\subsection{Compilation and generated code}

\textit{SE, JH:} Compile-time allocation of all data (exception: activation records)

\textit{SE, JH:} Facility for efficient code generation with compile-time
constants (should this be syntactically different?).

\textit{JH:} provide some FFI that supports callback, to interface with both
blocking and non-blocking functions.

\textit{JH:} provide some kind of immediately effectful print statement.

\textit{SE:} No need for separate compilation, just do whole-program
compilation. This makes monomorphisation straightforward/simple.

\subsection{IR}

\begin{itemize}
\item Variables and literals
\item Lambda abstractions
\item Function application
\item Pattern matching
\end{itemize}

\section{Overview}

\begin{lstlisting}
main(led : Ref (Sched Bool) ) =
  loop
    after 50 ms, led <- True
    wait led
    after 50 ms, led <- False
    wait led
\end{lstlisting}

\begin{lstlisting}
toggle(led : Ref (Sched Bool)) =
  led <- not led

slow(led : Ref (Sched Bool)) =
  let e1 = Occur : Sched Event
  loop
    toggle led
    after 30 ms, e1 <- Occur
    wait e1

fast(led : Ref (Sched Bool)) =
  let e2 = Occur : Sched Event
  loop
    toggle led
    after 20 ms, e2 <- Occur
    wait e2
    
main(led : Ref (Sched Bool)) =
  pipe slow led
       fast led
\end{lstlisting}

\section{Lexical Conventions}

\subsection{Comments}

Single line comments start with \lstinline!//! and end at a newline.

Multi-line comments start with \lstinline!/*! and end at a matching
\lstinline!*/!.  Multi-line comments nest and may also include
single-line comments.

\subsection{Indentation and Grouping}

\subsection{Lexer States}

The lexer state is maintained in a stack. Elements of the state stack may be one
of the following:

\begin{itemize}

\item Freeform: The scanner discards whitespace and does not insert
  grouping tokens

\item InBlock: The scanner is in a block; a separator is inserted in
  front of each line that starts at the current indentation level.  A
  line that starts to the left of the current indentation level causes
  a close-block token to be inserted (i.e., \texttt{\}}) and the
    current state to be popped.

\item StartBlock: The scanner will start a block (insert a \texttt{\}})
at the start of the next token and enter the InBlock state

\item StartBlockNL: The scanner will start a block (insert a
  \texttt{\}}) at the token that starts the next line and enter the InBlock
    state

\end{itemize}

\subsection{User-defined operators}

The parser naively parses every compound expression into a flat list of atomic
expression terminals. A later stage of the compiler uses the declared operator
precedences and reassembles the AST into a tree.

NOTE: Avoid Coq's mess of scope declarations and accompanying plumbing; take
a look at Agda's user-defined operators.

\section{Types}
sslang exposes two classes of types. \\

\noindent The first class of types encompasses primitive types: integers, booleans, pure events (i.e. conditional variable), and unit (i.e. void).\\ 

\noindent The second class of types are higher order types that are composed of primitive types:
\begin{itemize}
    \item \texttt{[<type>]}: An array containing values of type \texttt{<type>}
    \item \texttt{(<type>, ...)}: A tuple containing values of type \texttt{<type>}
    \item \texttt{<type1> $\rightarrow$ <type2>}: A function that takes a   \texttt{<type1>} and returns a  \texttt{<type2>}
    \item \texttt{\&<type>}: A reference to a variable of type \texttt{<type>}
    \item \texttt{Sched <type>}: A scheduled variable of type \texttt{<type>}. These variables are scheduled and managed by the SSM runtime and are used to synchronize routines
\end{itemize}

\noindent This is summarized by the following grammar:
\setlength{\grammarindent}{5em}
\begin{grammar}
<type> ::= [<type>]
\alt (<type>(, <type>)+)
\alt <type> $\rightarrow$ <type>
\alt \&<type>
\alt Sched <type>
\alt (<type>)
\alt Int8 | Int16 | Int32 | Int64
\alt UInt8 | UInt16 | UInt32 | UInt64
\alt Bool
\alt Event
\alt ()
\end{grammar}

\section{Routines}
\noindent Routines are defined at the top-level of a sslang file and consistent of the following components:
\begin{itemize}
    \item Routine name
    \item Zero or more parameters, each of which are optionally typed
    \item Optionally specified return type
    \item Routine body
\end{itemize}
\noindent The compiler will perform type inference for untyped components and will verify that typed components are correct.

\subsection{Routine Signatures}
\noindent sslang supports three ways of writing out type signatures for routines.
\begin{itemize}
    \item Infix-tuple: Type annotations are written immediately after identifiers.\\\\
    \texttt{foo (a: Int, b: Bool) $\rightarrow$ Int = \{...\}}\\\\
    This signature is reminiscent of Python's type signatures, which use colons for individual identifiers and then arrow for the return type.\\
    
    However, unlike Python, this type-signature actually indicates that \texttt{foo} only takes one argument, which is a tuple containing two \texttt{Int}s. If we wanted to specify that \texttt{foo} takes two \texttt{Int} arguments, we'd write the following:\\\\
    \texttt{foo (a: Int) (b: Bool) $\rightarrow$ Int = \{...\}}\\\\
    ...which is more reminiscent of OCaml type signature syntax.
    
    \item Postfix-tuple: Type annotations for tuples are written after the tuple, as in OCaml.\\\\
    \texttt{foo ((a, b): (Int, Bool)) $\rightarrow$ Int = \{...\}}
    \item Postfix-functional: Haskell-like type signature expressed after all parameters. For example, a function that takes two integers and returns a boolean:\\\\
    \texttt{foo x y: Int $\rightarrow$ Int $\rightarrow$ Bool = \{...\}}
\end{itemize}

\noindent These forms are summarized by the following grammar:
\setlength{\grammarindent}{5em}
\begin{grammar}
<FUNCDEF> ::= ID () ($\rightarrow$ <TYPE>)? = \{ <BODY> \}
\alt ID <PARAM_T>+ ($\rightarrow$ <TYPE>)? = \{ <BODY> \}
\alt ID <PARAM_T>+ (: <TYPE>)? = \{ <BODY> \}

<PARAM_T> ::= ID
\alt (<PARAM>)
\alt (<PARAM>(, <PARAM>)+)

<PARAM> ::= (<PARAM>)
\alt ID(: <TYPE>)?
\alt (<PARAM>(,<PARAM>)+)(: <TYPE>)?
\end{grammar}

\section{Expressions}

\end{document}
