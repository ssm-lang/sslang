\documentclass{article}
\usepackage{times}

\usepackage{listings}
\usepackage{xcolor}

\definecolor{listingbackground}{rgb}{0.8,0.8,0.8}

\title{SSLANG Language Reference Manual}
\author{Hans Montero, John Hui, and Stephen A. Edwards}
\date{2021}

\lstdefinelanguage{sslang}{
  morekeywords={let,if,then,else,while,after,wait,pipe},
  morecomment=[l]{//},
  morecomment=[s]{/*}{*/},
  columns=flexible,
}

\lstset{language=sslang,
  columns=flexible,
  commentstyle={\itshape\color{red}},
  backgroundcolor=\color{listingbackground},
}

\begin{document}

\maketitle

\section{Overview}

\begin{lstlisting}
main(led : Ref (Sched Bool) ) =
  loop
    after 50 ms, led <- True
    wait led
    after 50 ms, led <- False
    wait led
\end{lstlisting}

\begin{lstlisting}
toggle(led : Ref (Sched Bool)) =
  led <- not led

slow(led : Ref (Sched Bool)) =
  let e1 = Occur : Sched Event
  loop
    toggle led
    after 30 ms, e1 <- Occur
    wait e1

fast(led : Ref (Sched Bool)) =
  let e2 = Occur : Sched Event
  loop
    toggle led
    after 20 ms, e2 <- Occur
    wait e2
    
main(led : Ref (Sched Bool)) =
  pipe slow led
       fast led
\end{lstlisting}

\section{Lexical Conventions}

\subsection{Comments}

Single line comments start with \lstinline!//! and end at a newline.

Multi-line comments start with \lstinline!/*! and end at a matching
\lstinline!*/!.  Multi-line comments nest and may also include
single-line comments.

\subsection{Indentation and Grouping}

\subsection{Lexer States}

\begin{itemize}

\item Freeform: The scanner discards whitespace and does not insert
  grouping tokens
  
\item InBlock: The scanner is in a block; a separator is inserted in
  front of each line that starts at the current indentation level.  A
  line that starts to the left of the current indentation level causes
  a close-block token to be inserted (i.e., \lstinline!}!) and the
    current state to be popped.
    
\item StartBlock: The scanner will start a block (insert a
  \lstinline!{!) at the start of the next token and enter the InBlock
    state
    
\item StartBlockNL: The scanner will start a block (insert a
  \lstinline!{!) at the token that starts the next line and enter the InBlock
    state

\end{itemize}

\section{Types}

\section{Expressions}

\end{document}
