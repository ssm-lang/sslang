\documentclass{article}
\usepackage{times}

\usepackage{listings}
\usepackage{xcolor}
\usepackage{hyperref}
\usepackage{syntax}
\renewcommand{\grammarlabel}[2]{\textit{#1}\hfill#2}
\renewcommand{\syntleft}{\itshape\normalsize}
\renewcommand{\syntright}{\normalfont\normalsize}
\renewcommand{\ulitleft}{\normalfont\ttfamily\small}
\renewcommand{\ulitright}{\normalsize\normalfont}
\definecolor{listingbackground}{rgb}{0.8,0.8,0.8}

\title{SSLANG Language Reference Manual}
\author{Hans Montero, John Hui, and Stephen A. Edwards}
\date{2021}

\lstdefinelanguage{sslang}{
  morekeywords={let,if,then,else,while,after,wait,pipe},
  morecomment=[l]{//},
  morecomment=[s]{/*}{*/},
  columns=flexible,
}

\lstset{language=sslang,
  columns=flexible,
  commentstyle={\itshape\color{red}},
  backgroundcolor=\color{listingbackground},
}

\begin{document}

\maketitle

\section{Wish List}

\subsection{Philosphy}

\textit{SE:} Make the IR, especially, as simple as possible (a la
GHC). Take a Reynolds-like approach where the larger language has more
features that boil down into simpler primitives.  E.g., typeclasses
can be expressed as dictionaries with function pointers and completely
evaporate.  I'm inspired by the make-it-explicit approach to garbage
collection operations in the Perceus tech report (Reinking et al.,
MSR-TR-2020-42).

\subsection{Syntax}

\textit{SE:} Minimal punctuation (e.g., print ``hello'' vs. print(``hello'')).
\textit{JH:} I would like to have punctuation for types/operations with special
built-in meaning, e.g., \verb|&| or \verb|*| for references. I'm not opposed to
using parens for function application, but it depends on whether we can have
partial application.

\textit{SE, JH:} Nested comments

\textit{JH:} Single-line comment syntax. I think this makes sense especially for
a whitespace-sensitive language.

\textit{SE, JH:} Indentation-sensitive nesting; keywords start blocks (Haskell-style).

\textit{SE, JH:} User-defined binary operators and precedence levels.

\textit{JH:} User-defined type-level operator definitions. e.g., \verb|$| for
avoiding parens in type application, \verb|&| for ref.

\textit{JH:} some kind of readable, forward-thinking syntax for annotations that
the compiler can take advantage of, e.g., must-inline, dont-inline,
stack-allocate, maybe-unused. Should be less hacky than Haskell's
\verb|{-# LANGUAGE ... #-}| and C/C++'s \verb|#pragma ..| and
\verb|__attribute__((..))|.

\textit{SE:} C\# has an elaborate attributes mechanism.
Most declarations allow attributes to be placed in square brackets before.  \url{https://docs.microsoft.com/en-us/dotnet/csharp/programming-guide/concepts/attributes/}

\begin{lstlisting}{language=[Sharp]C}
  
[Serializable]
public class SampleClass
{
    // Objects of this type can be serialized.
} 
\end{lstlisting}

\subsubsection{Discussions}

\subsection{Types and expressions}

\textit{SE, JH:} Type expressions.

\textit{SE, JH:} Algebraic data types + pattern-matching.

Statically sized arrays passable through abstract array arguments, e.g., so you
can write a sort that's polymorphic in the size of the array; implies array size
is a runtime value, not (just) compile-time.

\textit{SE, JH:} Typeclasses at least for arithmetic operators

Pass-by-value and pass-by-reference values

\textit{JH:} References to arrays of T, arrays of arrays of T

\textit{JH, SE:} Array slices (``safe'' references to parts of arrays that play
well with parallelism)

\textit{SE, JH:} Type inference at least for local variables

\textit{JH:} Type annotations for function arguments

\textit{SE, JH:} Parametric polymorphism, likely eliminated by whole-program
monomorphisation (type variables)

\textit{SE, JH:} A notion of constants (assign once) versus variables (assign
multiple times)

\textit{JH:} a notion of ownership that users can opt into (which reduces to
reference-sharing otherwise).

\subsubsection{Discussions}

Partial application?  Or do we prohibit it?

\subsection{Control flow}

\textit{SE, JH:} Parallel function calls.

\textit{SE, JH:} Wait on variables

\textit{SE, JH:} Immediate and delayed assignment ``primitives''.

\textit{JH:} Some sort of exception handling, which supports missed deadlines.

\textit{SE, JH:} combinators to derive "wait until" and "wait both" and "wait either".

\subsection{Compilation and generated code}

\textit{SE, JH:} Compile-time allocation of all data (exception: activation records)

\textit{SE, JH:} Facility for efficient code generation with compile-time
constants (should this be syntactically different?).

\textit{JH:} provide some FFI that supports callback, to interface with both
blocking and non-blocking functions.

\textit{JH:} provide some kind of immediately effectful print statement.

\textit{SE:} No need for separate compilation, just do whole-program
compilation. This makes monomorphisation straightforward/simple.

\subsection{IR}

\begin{itemize}
\item Variables and literals
\item Lambda abstractions
\item Function application
\item Pattern matching
\end{itemize}

\section{Overview}

\begin{lstlisting}
main(led : Ref (Sched Bool) ) =
  loop
    after 50 ms, led <- True
    wait led
    after 50 ms, led <- False
    wait led
\end{lstlisting}

\begin{lstlisting}
toggle(led : Ref (Sched Bool)) =
  led <- not led

slow(led : Ref (Sched Bool)) =
  let e1 = Occur : Sched Event
  loop
    toggle led
    after 30 ms, e1 <- Occur
    wait e1

fast(led : Ref (Sched Bool)) =
  let e2 = Occur : Sched Event
  loop
    toggle led
    after 20 ms, e2 <- Occur
    wait e2
    
main(led : Ref (Sched Bool)) =
  pipe slow led
       fast led
\end{lstlisting}

\section{Lexical Conventions}

\subsection{Comments}

Single line comments start with \lstinline!//! and end at a newline.

Multi-line comments start with \lstinline!/*! and end at a matching
\lstinline!*/!.  Multi-line comments nest and may also include
single-line comments.

\subsection{Indentation and Grouping}

\subsection{Lexer States}

The lexer state is maintained in a stack. Elements of the state stack may be one
of the following:

\begin{itemize}

\item Freeform: The scanner discards whitespace and does not insert
  grouping tokens

\item InBlock: The scanner is in a block; a separator is inserted in
  front of each line that starts at the current indentation level.  A
  line that starts to the left of the current indentation level causes
  a close-block token to be inserted (i.e., \texttt{\}}) and the
    current state to be popped.

\item StartBlock: The scanner will start a block (insert a \texttt{\}})
at the start of the next token and enter the InBlock state
\item StartBlockNL: The scanner will start a block (insert a
  \texttt{\}}) at the token that starts the next line and enter the InBlock
    state

\end{itemize}

\subsection{User-defined operators}

The parser naively parses every compound expression into a flat list of atomic
expression terminals. A later stage of the compiler uses the declared operator
precedences and reassembles the AST into a tree.

NOTE: Avoid Coq's mess of scope declarations and accompanying plumbing; take
a look at Agda's user-defined operators.

\section{Types}
\noindent sslang's type system is composed of two fundamental components: type constructors and type application. All types in sslang are a combination of type constructors and applications. \\

\noindent Type constructors may take zero or more type arguments that, when composed via type application, form a fully-qualified type. \\ 

\noindent Constructors that take zero type arguments can be thought of as ``primitive types", such as integers, booleans, pure events (i.e. conditional variable), and unit (i.e. void). \\

\noindent Constructors that take more than type argument can be thought of as ``higher-order types". Here's a summary of sslang's higher-order types:
\begin{itemize}
    \item \texttt{[type]}: An array containing values of type \texttt{type}
    \item \texttt{(type, ...)}: A tuple containing values of type \texttt{type}
    \item \texttt{type1 -> type2}: A function that takes a   \texttt{type1} and returns a  \texttt{type2}
    \item \texttt{\&type}: A reference to a variable of type \texttt{type}
    \item \texttt{Sched type}: A scheduled variable of type \texttt{type}. These variables are scheduled and managed by the SSM runtime and are used to synchronize routines
\end{itemize}

\noindent This all is summarized by the following grammar:
\setlength{\grammarindent}{5em}
\begin{grammar}
<type> ::= "("<type>")"
\alt "["<type>"]"
\alt "("<type>(","<type>)$^{+}$")"
\alt <type> "->" <type>
\alt "\&"<type>
\alt "()"
\alt "type-id" <type>$^{*}$
\end{grammar}
\noindent where \texttt{type-id ::= [A-Z][a-zA-Z0-9]$^{*}$}\\

\noindent Note the generic definition of type constructors: simply a type identifier followed by one or more \textit{type}s. This implementation is meant to facilitate support for user-defined types and to provide a concise interface for sslang's type system to deal with.
\section{Routines and Top-Level Definitions}
\noindent Top-level definitions in sslang consistent of the following components:
\begin{itemize}
    \item Identifier name (global variable or routine name)
    \item Zero or more parameters, possibly type-annotated
    \item Annotated return type
    \item Body
\end{itemize}
\noindent The compiler will perform type inference for untyped components and will verify that typed components are correct.

\subsection{Type Signatures}
\noindent At the core of sslang routine signatures, we have two symbols that provide type information about a language construct. Their meanings are as follows:
\begin{itemize}
    \item \texttt{:} (Colon): ``is of type"
    \item \texttt{->} (Right Arrow): ``returns"
\end{itemize}

\noindent We use these symbols to separate type signatures into two broad syntactic categories:
\begin{itemize}
    \item \texttt{foo <params> -> <ret-type>}\\
          Using the ``\texttt{->}" syntax, the return type of the routine is denoted at the end of the signature. This is evocative of languages like Python and Rust.
    \item \texttt{foo <untyped-params> : <fn-type>}\\
          Using the ``\texttt{:}" syntax, we opt for annotating the type of the entire routine at the end of the signature instead of typing the parameters individually and the return type separately. This is evocative of Haskell's type signatures. If no parameters are specified, the definition is interpreted as a global variable.
\end{itemize}

\noindent When annotating parameters, sslang uses the ``\texttt{:}" syntax to specify the type of a parameter. sslang also supports distributing type annotations over tuple members. As such, the following two type annotations are equivalent:
\begin{itemize}
    \item \texttt{(a: Int, b: Bool)}
    \item \texttt{(a , b) : (Int, Bool)}
\end{itemize}

\noindent Note that while we support these two styles of tuple type annotation, only one form may be used. All identifiers must be type annotated at most once. \\

\noindent Top-level definitions can be summarized by the following grammar:
\setlength{\grammarindent}{9em}
\begin{grammar}
<topdef> ::= "id" <param_t>$^{+}$ ("->" <type>)$^{?}$ "= \{" <body> "\}"
\alt "id" <param_t>$^{*}$ ":" <type> "= \{" <body> "\}"

<param_t> ::= <param_atomic>
\alt "("<param>")"
\alt "("<param>("," <param>)$^{+}$")"

<param> ::= "("<param>")"
\alt <param_atomic> (":" <type>)$^{?}$
\alt "("<param>(","<param>)$^{+}$")"(":" <type>)$^{?}$

<param_atomic> ::= "id"
\alt "()"
\alt "_"
\end{grammar}

\section{Expressions}

\end{document}