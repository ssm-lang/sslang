\documentclass{article}
\usepackage{times}

\usepackage{listings}
\usepackage{xcolor}
\usepackage{hyperref}
\usepackage{syntax}
\usepackage{amsmath}
\usepackage{algorithm2e}
\renewcommand{\grammarlabel}[2]{\textit{#1}\hfill#2}
\renewcommand{\syntleft}{\itshape\normalsize}
\renewcommand{\syntright}{\normalfont\normalsize}
\renewcommand{\ulitleft}{\normalfont\ttfamily\bfseries}
\renewcommand{\ulitright}{\normalsize\normalfont}
\def\<#1>{\synt{#1}}
\definecolor{listingbackground}{rgb}{0.8,0.8,0.8}
\definecolor{commentcolor}{rgb}{0.2,0.2,0.8}

\title{SSLANG Language Reference Manual}
\author{Hans Montero, John Hui, and Stephen A. Edwards}
\date{2021}

\lstdefinelanguage{sslang}{
  morekeywords={let,if,else,while,after,wait,fun,do,loop,match,par},
  morecomment=[l]{//},
  morecomment=[s]{/*}{*/},
  columns=flexible,
}

\lstset{language=sslang,
  columns=flexible,
  commentstyle={\itshape\color{commentcolor}},
  backgroundcolor=\color{listingbackground},
}

\newcommand{\fixme}[1]{
  \noindent\colorbox{yellow}{\parbox{\dimexpr\linewidth-2\fboxsep}{#1}}%
}

% Turn off paragraph indenting
\parindent=0pt
% Add space between paragraphs
\parskip=0.5\baselineskip

\begin{document}

\maketitle

\section{Documentation Conventions}

In the grammar that follows, literal terminals are printed \lit*{bold
  mono spaced}.  Syntactic categories are written in \<italics>.
Normal parentheses () are used for grouping, superscripts represent
$^*$ zero or more and $^+$ one or more, and square brackets []
indicate character ranges, e.g., [\lit*{A}--\lit*{Z}] indicates all
uppercase letters.

\section{Lexical Conventions}

\subsection{Comments}

Single line comments start with \lit*{//} and end at a newline.

Multi-line comments start with \lit*{/*} and end at a matching
\lit*{*/}.  Multi-line comments nest and may also include
single-line comments.

To avoid confusion with comments, user-defined operators
should not contain \lit*{//} or \lit*{/*} as substrings.
Furthermore, comments must be separated from operator symbols
by at least one character:

\begin{lstlisting}
&// invalid comment
&/* invalid comment */
& // valid comment
& /* valid comment */
\end{lstlisting}

\subsection{Layout}

Sslang's syntax contains \emph{blocks} surrounded by brace tokens \lit*{\{} and \lit*{\}}.
Blocks consist of other syntactic items separated by some \emph{separator} token.

For brevity, sslang allows the programmer to selectively omit braces \lit*{\{} and \lit*{\}}
as well as certain separator tokens, and instead convey the same information
using \emph{layout} (indentation).

Layout-sensitive and layout-insensitive syntax are both supported and
mutually compatible, so the meaning of a layout-sensitive program can always
be specified by adding braces and tokens to produce some equivalent
layout-insensitive program.
This section describes the procedure by which those tokens are inserted.

\subsubsection{Implicit Blocks}

Items in sslang's syntax that are surrounded by braces are called \emph{blocks},
and the enclosed items of a block are delimited by some \emph{separator} token (e.g., \lit*{;}).
We call blocks whose tokens are inserted by the scanner \emph{implicit} blocks.

In an implicit block, items all share the same indentation; visually, they are
vertically aligned.
The scanner treats lines with greater indentation
(that are not part of an inner block) as part of the previous item;
these are called \emph{line continuations}.
When the scanner comes across lines with less indentation,
an implicit block is ended.

For instance, a user may write:

\begin{lstlisting}
loop
  f x
    y
  g x
h x
\end{lstlisting}

The scanner transforms this input into:

\begin{lstlisting}
loop
  { f x y
  ; g x
  }
; h x
\end{lstlisting}

Implicit blocks are started after \emph{layout} keywords, which come in two flavors:
\emph{layout-next-token} keywords, which start blocks at the next token, and
\emph{layout-next-line} keywords, which start blocks on the following line.

An example of a \emph{layout-next-line} keyword is \lit*{if}:

\begin{lstlisting}
if f x
  g x
  g y
\end{lstlisting}

which is expanded to:

\begin{lstlisting}
if f x
  { g x
  ; g y
  }
\end{lstlisting}

Sslang recognizes the following \emph{layout-next-token} keywords:

\begin{itemize}
  \item \lit*{loop} with \lit*{;} separating sequenced expressions
  \item \lit*{=} with \lit*{;} separating sequenced expressions
  \item \lit*{else} with \lit*{;} separating sequenced expression
  \item \lit*{wait} with \lit*{||} separating sensitized scheduled variables
  \item \lit*{par} with \lit*{||} separating parallel expressions
  \item \lit*{let} with \lit*{||} separating mutually recursive definitions
\end{itemize}

And the following \emph{layout-next-line} keywords:

\begin{itemize}
  \item \lit*{if} with \lit*{;} separating sequenced expressions
  \item \lit*{while} with \lit*{;} separating sequenced expressions
  \item \lit*{match} with \lit*{|} separating match arms
\end{itemize}

\subsubsection{Explicit Blocks}

Programmers may also choose to write braces and separators anyway,
where allowed by the syntax; these blocks are called \emph{explicit} blocks.
Writing explicit blocks effectively turns off the scanner's token insertion for
that block (though inner implicit blocks may still be created).

For instance, if the user writes:

\begin{lstlisting}
loop {
  f x;
  wait x
       y
}
\end{lstlisting}

The scanner only encloses tokens in the implicit block introduced by \lit*{wait}:

\begin{lstlisting}
loop {
  f x;
  wait {  x
       || y
       }
}
\end{lstlisting}

Inside of an explicit block, the scanner does not insert separators and ignores
indentation (unless an inner block is started). So the following is legal:

\begin{lstlisting}
loop
  loop {
f x
;
g y
}
h z
\end{lstlisting}

And is expanded into the following

\begin{lstlisting}
loop {
  loop {
    f x ;
    g y
  }
};
h z
\end{lstlisting}

\subsubsection{Matching Delimiters}

In addition to the braces \lit*{\{} \lit*{\}}, which surround blocks,
sslang also includes other matched delimiters: \lit*{[} \lit*{]} and \lit*{(} \lit*{)}.
To ensure that any program with matched delimiters remains correctly matched
after token insertion, the scanner also keeps track of these delimiter pairs.
In particular, if it comes across an explicit closing delimiter, it closes all
implicit blocks opened since the corresponding opening delimiter, even if
indentation would not have otherwise caused it to do so.

For instance, the following:

\begin{lstlisting}
(loop
   wait x)
\end{lstlisting}

Is expanded as:

\begin{lstlisting}
(loop
  { wait {x}})
\end{lstlisting}

Rather than:

\begin{lstlisting}
(loop
  { (wait {x)}
  }
\end{lstlisting}

Note, however, that it makes no such consideration for separator tokens. So the following:

\begin{lstlisting}
loop
  wait x;
  f x
\end{lstlisting}

is expanded into:

\begin{lstlisting}
loop
  { wait {x;}
  ; f x
  }
\end{lstlisting}

which the parser shall deem a syntax error.

\subsubsection{Layout Algorithm}

This section describes the layout algorithm the sslang scanner uses
to insert braces and separators prior to parsing.
The pseudocode in Algorithm~\ref{alg:layout-algorithm} receives tokens from the
source program as input, and emits zero or more output tokens for each input
token.

The input token stream consists of only non-whitespace tokens and newline tokens.
The algorithm assumes that indentation information conveyed by horizontal
whitespace (spaces and tabs) is attached to all non-whitespace tokens.
The indentation of any token $t$ is a natural number indicating the column in
the source text where the first character of $t$ appears, and is denoted by
$\Delta(t)$.

The scanner maintains a stack of states, where $\sigma_n$ denotes the $n$th
state from the top of the stack;
$\sigma_0$ is the state at the top of the stack.
The scanner updates $\sigma_0$ by pushing and popping states to and from the
state stack, following a \emph{first-in-first-out} discipline.
Each state in the state stack falls into one of three categories $\{ B, I, N \}$:

\begin{itemize}
  \item $B$ means that the scanner is within an explicit block.
    $B$ states remember the matching delimiter that ends that block.
    For example, in an explicit block started by \lit*{(}, the matching delimiter is \lit*{)}.

  \item $I$ means that the scanner is within an implicit block.
    $I$ states remember the separator that separates items of that block, as
    well as the indentation level of that block.
    For example, in an implicit block started by a \lit*{par}, $\sigma_0$ is $B$
    with separator \lit*{||} and the indentation of the first token following the \lit*{par}.

  \item $N$ means that the scanner is scanning through the rest of a line
    following a \emph{layout-next-line} keyword.
    $N$ states remember the separator of the \emph{layout-next-line} keyword.
    For example, following an \lit*{if}, $\sigma_0$ is $N$ with separator \lit*{;}.
\end{itemize}

The ``current indentation'' of the scanner is written $\Delta(\sigma_0)$, computed by:

\begin{equation*}
  \Delta(\sigma_n) =
  \begin{cases}
    i                     & \text{ $\sigma_n$ is $I$ with indentation $i$ } \\
    \Delta(\sigma_{n+1})  & \text{otherwise}
  \end{cases}
\end{equation*}

\begin{algorithm}[H]

\newcommand{\commentfont}{\relsize{-1}\color{commentcolor}\it}
\SetCommentSty{commentfont}
\SetFuncSty{bf}
\SetArgSty{}

% Pythonic pseudocode, get rid of clutter from do, end
\SetStartEndCondition{ }{}{}%
\SetKwIF{If}{ElseIf}{Else}{if}{:}{else if}{else:}{}%
\SetKwBlock{Case}{case}{:}%
\SetKwSwitch{Switch}{Case}{Other}{switch}{:}{case}{otherwise}{}{}%
\SetKwFor{While}{while}{:}{}

\DontPrintSemicolon
\SetAlgoVlined
% \SetAlgoNoEnd

\SetKwData{CurrentState}{$\sigma_0$}
\SetKwFunction{Indent}{$\Delta$}
\SetKwData{CurrentIndent}{\Indent{$\sigma_0$}}

Initialize \CurrentState to $I$ with separator \lit{||} and indentation 1 \;
\Repeat{end of file has been reached}{
  Scan to and consume next non-whitespace token $t$ \;
  \uCase{$t$ is a newline}{
    Scan to (but don't consume) next non-whitespace token $t'$ \;
    \While (\tcc*[f]{$t'$ closes an implicit block})
    {\CurrentState is $I$ and \Indent{$t'$} $<$ \CurrentIndent} {
      Pop state and emit \lit{\}} \;
    }
    \uIf (\tcc*[f]{$t'$ starts a new implicit block item})
    {\CurrentState is $I$ and \Indent{$t'$} $=$ \CurrentIndent } {
      Emit separator of \CurrentState \;
    }
    \uElseIf (\tcc*[f]{$t'$ is a line continuation})
    {\CurrentState is $I$ and \Indent{$t'$} $>$ \CurrentIndent } {
      Do nothing \;
    }
    \uElseIf (\tcc*[f]{$t'$ starts a \emph{layout-next-line} block})
    {\CurrentState is $N$ with separator $s$} {
      Assert \Indent{$t'$} $>$ \CurrentIndent \;
      Pop state and emit \lit{\{} \;
      Push $I$ with separator $s$ and indentation \Indent{$t'$} \;
    }
    \ElseIf (\tcc*[f]{$t'$ starts a line in an explicit block})
    {current state is $B$} {
      Do nothing \;
    }
  }
  \uCase{$t$ is \lit{(}, \lit{[}, or \lit{\{}} {
    \If (\tcc*[f]{$t'$ explicitly starts a \emph{layout-next-line} block})
    {$t$ is \lit{\{} and \CurrentState is $N$} {
      Pop state \;
    }
    Push $B$ with matching delimiter \lit{)}, \lit{]}, or \lit{\}} \;
    Emit $t$ \;
  }
  \uCase{$t$ is \lit{)}, \lit{]}, or \lit{\}}} {
    \While (\tcc*[f]{implicit block needs to be closed})
    {\CurrentState is $I$} {
      Pop state and emit \lit{\}} \;
    }
    Assert \CurrentState is $B$ with matching delimiter $t$ \;
    Pop state and emit $t$ \;
    % B with non-matching delimiter means something like (}
    % Otherwise means state was N, which makes no sense
  }
  \uCase(\tcc*[f]{e.g., \lit*{loop}}) {$t$ is a \emph{layout-next-token} keyword} {
    Scan to and consume next non-whitespace token $t'$ \;
    \uIf(\tcc*[f]{$t'$ starts an explicit block}) {$t'$ is \lit{\{}}{
      Push $B$ with matching delimiter \lit{\}} \;
      Emit $t$ \lit{\{} \;
    }\Else (\tcc*[f]{$t'$ starts an implicit block}) {
      Assert \Indent{$t'$} $>$ \CurrentIndent \;
      Push $I$ with separator of $t$ and indentation \Indent{$t'$} \;
      Emit $t$ \lit{\{} $t'$ \;
    }
  }
  \uCase(\tcc*[f]{e.g., \lit*{if}}) {$t$ is a \emph{layout-next-line} keyword} {
    Push $N$ with separator of $t$ \;
    Emit $t$ \;
  }
  \Other(\tcc*[f]{$t$ is any plain old token with no implications for layout}){
    Emit $t$ \;
  }
}
\caption{The layout algorithm.}
\label{alg:layout-algorithm}
\end{algorithm}

\subsection{User-defined operators}

The parser naively parses every compound expression into a flat list of atomic
expression terminals. A later stage of the compiler uses the declared operator
precedences and reassembles the AST into a tree.

\fixme{Avoid Coq's mess of scope declarations and accompanying plumbing; take
a look at Agda's user-defined operators.}

\section{Types}

\setlength{\grammarindent}{6em}
\begin{grammar}
<type> ::= <type-app> "->" <type>
\alt <type-app>

<type-app> ::= <type-app> <type-atom>
\alt <type-atom>

<type-pre> ::= "&" <type-pre>
\alt <type-atom>

<type-atom> ::= <type-id>
\alt "(" ")"
\alt "[" <type> "]"
\alt "(" <type> ("," <type>)$^*$ ")"

<type-id> ::= ["A"--"Z"]["a"--"zA"--"Z0"--"9"]$^*$
\end{grammar}

\fixme{What about type variables?  Are we going to use OCaml-style \texttt{'a}
  or just leave them as type-ids?}

Type names start with a capital letter followed by zero or more
letters and digits.

A type of the form \<type>$_1$ \lit*{->} \<type>$_2$ represents a function
from \<type>$_1$ to \<type>$_2$. Also, \lit*{->} groups
right-to-left, e.g.,
\lit*{Int} \lit*{->} \lit*{Int} \lit*{->} \lit*{Int}
means
\lit*{Int} \lit*{->} \lit*{(} \lit*{Int} \lit*{->} \lit*{Int} \lit*{)}.

Types may be polymorphic and take one or more types as arguments.
Type argument application is denoted by juxtaposition and groups
left-to-right, e.g.,
\lit*{Either} \lit*{Int} \lit*{Char}
means
\lit*{(} \lit*{Either} \lit*{Int} \lit*{)} \lit*{Char}.

Parentheses provide a way to override precedence; empty parentheses
\lit*{()} denote the ``unit'' type.

A type in square brackets indicates a list of that type, e.g.,
\lit*{[} \lit*{Int} \lit*{]} is a list of integers.

An ampersand \lit*{\&} before a type indicates a reference to that type.

\fixme{Is this a mutable reference, or is that something different?}

A parentheses-enclosed comma-separated list of two or more types
denotes a tuple, e.g., \lit*{(} \lit*{Int} \lit*{,} \lit*{Int}
\lit*{)} is a pair of integers.

\section{sslang Top-Level}
The top-level of sslang source code consists of a series of definitions. These definitions are either function definitions or global variable definitions. They are composed from the following:
\begin{itemize}
    \item Identifier name (global variable or function name)
    \item At least one pattern defining parameters for functions. If none, then the definition is for a global variable
    \item Annotated return type
    \item Body
\end{itemize}

Patterns can be simple identifiers, wildcards, tuples, or a mixture of both to deconstruct algebraic data types.
\subsection{Type Signatures}
At the core of sslang type signatures, we have two symbols that provide type information about a language construct. Their meanings are as follows:
\begin{itemize}
    \item \texttt{:} (Colon): ``is of type"
    \item \texttt{->} (Right Arrow): ``returns"
\end{itemize}

We use these symbols to separate type signatures into two broad syntactic categories:
\begin{itemize}
    \item \texttt{foo <params> -> <ret-type>}\\
          Using the ``\texttt{->}" syntax, the return type of the function is denoted at the end of the signature. This is evocative of languages like Python and Rust.
    \item \texttt{foo <untyped-params> : <fn-type>}\\
          Using the ``\texttt{:}" syntax, we opt for annotating the type of the entire function at the end of the signature instead of typing the parameters individually and the return type separately. This is evocative of Haskell's type signatures. If no parameters are specified, the definition is interpreted as a global variable.
\end{itemize}

When annotating parameters, sslang uses the ``\texttt{:}" syntax to specify the type of a parameter. Note that type annotation is only possible within parenthesis due to the lower precedence of ``\texttt{:}".

sslang also supports distributing type annotations over tuple members. As such, the following two type annotations are equivalent:
\begin{itemize}
    \item \texttt{(a: Int, b: Bool)}
    \item \texttt{(a , b) : (Int, Bool)}
\end{itemize}

While we support these two styles of tuple type annotation, only one form may be used. All identifiers must be type annotated at most once. \\

Top-level definitions can be summarized by the following grammar.
\setlength{\grammarindent}{9em}
\begin{grammar}
<def_let> ::= <pat>$^{+}$ <typ_fn> "= \{" <body> "\}"

<typ_fn> ::= "->" <type>
\alt ":" <type>

<pat> ::= "id @" <pat>
\alt <pat_atomic>

<pat_atomic> ::= "id"
\alt "_"
\alt "()"
\alt "(" <pat_ann> ("," <pat_ann>)$^{*}$ ")"

<pat_ann> ::= <pat> ":" <typ>$^{?}$
\end{grammar}

\section{Expressions}
We now a present an overview of expressions and language constructs in sslang.
\begin{grammar}
<expr> ::= <expr_stm> ";" <expr>
\alt "let {" <def_let> ("||" <def_let>)$^{+}$ "}; " <expr>
\alt <expr_stm>

<expr_stm> ::= <expr_ann> "<- {"  <expr_ann> "}"
\alt "after" <expr_ann> "," <expr_ann> "<- {"  <expr_ann> "}"
\alt <expr_ann>

<expr_ann> ::= <expr_ann> ":" <typ>
\alt <expr_op>

<expr_op> ::= <expr_blk> <expr_op_region>$^{?}$

<expr_blk> ::=  "do {" <expr> "}"
\alt "loop {" <expr> "}"
\alt "wait {" <expr> ("||" <expr>)$^{+}$ "}"
\alt "par {" <expr> ("||" <expr>)$^{+}$ "}"
\alt "if" <expr_blk> "{" <expr> "}" <expr_else>$^{?}$
\alt "while" <expr_blk> "{" <expr> "}"
\alt "fun" <pat>$^{+}$ "{" expr "}"
\alt "match" <expr_blk> "{" (  <pat> "=>" "{" <expr> "}" )$^{+}$ "}"
\alt <expr_app>

<expr_else> ::= "; else {" <expr> "}"
\alt "else {" <expr> "}"

<expr_app> ::= <expr_app> <expr_atom>
\alt <expr_atom>

<expr_atom> ::= ["True"|"False"] \texttt{// Bool}
\alt ["0"-"9"]$^{+}$ \texttt{// Int}
\alt ["0"-"9"]$^{+}$"."["0"-"9"]$^{*}$ \texttt{// Rational}
\alt "()" \texttt{// Event}
\alt "'"["a"-"zA"-"Z"]"'" \texttt{// Char}
\alt "\""["a"-"zA"-"Z"]$^{+}$"\"" \texttt{// String}
\alt "(" <expr> ")" \texttt{// Parenthesized expression}
\alt ["a"-"zA"-"Z"] ["a"-"zA"-"Z0"-"9_'"]$^{*}$ \texttt{// Identfier}
\end{grammar}
\subsection{Atomic expressions}
There are no special rules for referencing variables or literals.

The following expression computes the sum of the variable \texttt{x} (defined elsewhere) and 5:
\begin{lstlisting}
x + 5
\end{lstlisting}
% add variables, add parenthesized exprs
Variables in sslang are simply identifiers composed of alphanumeric characters, underscores, and ticks: \texttt{[a-zA-Z] [a-zA-Z0-9_']*}

There are several literal values for sslang built-in types:

\begin{grammar}
<expr_atom> ::= ["True"|"False"] \texttt{// Bool}
\alt ["0"-"9"]$^{+}$ \texttt{// Int}
\alt ["0"-"9"]$^{+}$"."["0"-"9"]$^{*}$ \texttt{// Rational}
\alt "()" \texttt{// Event}
\alt "'"["a"-"zA"-"Z"]"'" \texttt{// Char}
\alt "\""["a"-"zA"-"Z"]$^{+}$"\"" \texttt{// String}
\alt "(" <expr> ")" \texttt{// Parenthesized expression}
\alt ["a"-"zA"-"Z"] ["a"-"zA"-"Z0"-"9_'"]$^{*}$ \texttt{// Identfier}
\end{grammar}
\subsection{Application}
\begin{grammar}
<app> ::= <expr_atom>$^{+}$

<expr_atom> ::= <literal>
\alt "(" <expr> ")"
\end{grammar}

Application in sslang is specified by juxtaposition of at least two expressions. In the following example, the expressions \texttt{x + 5} and \texttt{y} are applied to \texttt{f}, where \texttt{f} could be a data type constructor or a function:
\begin{lstlisting}
f (x + 5) y
\end{lstlisting}

Application via juxtaposition can be performed on adjacent literal expressions and parenthesized complex expressions, as shown in the grammar above.

\subsection{Operation Regions}
\fixme{todo}
\subsection{Do-blocks}
\begin{grammar}
<do> ::= "do {" <expr> "}"
\end{grammar}
Do-blocks allow for grouping of expressions without creating a new scope. This is reminiscent of parenthesizing complex expressions.
The value of do-block expressions is unit.
\subsection{Loops}
\begin{grammar}
<loop> ::= "loop {" <expr> "}"
\alt "while" <expr> "{" <expr> "}"
\end{grammar}
sslang loops come in two flavors: unconditional and conditional loops. The former is simply a sequence of expressions that are evaluated until a \texttt{break} expression is encountered The latter is simply while loop that evaluates the sequence of expressions in its body until its condition evaluates to \texttt{False}. The following two code snippets demonstrate how to count to five using these two loop constructs:
\begin{lstlisting}
x <- 0
loop
  if x == 5
    break
  else 
    x <- x + 1
    
y <- 0
while y != 5
  y <- y + 1
 
\end{lstlisting}
The value of loop and while expressions is unit.
\subsection{If-Else}
\begin{grammar}
<if-else> ::= "if" <expr> "{" <expr> "}" ("else" "{" <expr> "}")$^{?}$
\end{grammar}
If-else expressions conditionally evaluate blocks. If the predicate expression evaluates to \texttt{True}, the \texttt{if} block is evaluated. If there is an \texttt{else} block provided and the predicate evaluated to \texttt{False}, then the \texttt{else} block is evaluated.

Note that If-else creates indentation blocks. This is shown in the following example:
\begin{lstlisting}
if x > 5
    x <- x * 2
    f x
else
    x <- 0
\end{lstlisting}

The value of an if-else expression is the last expression in either of the blocks. The types of the last expressions in the \texttt{if} block and the \texttt{else} block  must be the same. When there is no \texttt{else} block provided, the \texttt{if} block must evaluate to a unit type, which means the value of the entire construct is a unit type.
\subsection{Lambdas}
\begin{grammar}
<lambda> ::= "fun" <pat>$^{+}$ "{" <expr> "}"
\end{grammar}

Lambda expressions enable programmers to define unnamed functions on-the-fly and use them immediately in other expressions without having to make a top-level definition.

Note that lambdas may take any number of pattern arguments, which allow for deconstruction of data types or specification of "don't-care" arguments with wildcards.

The following example demonstrates two ways of defining the same lambda, one using the syntactic sugar let-style and the other using the \texttt{fun} keyword:
\begin{lstlisting}
let f x y = x + y;
f 1 2 == (fun a b { a + b }) 1 2
\end{lstlisting}
\subsection{Parallel Evaluation}
\begin{grammar}
<par> ::= "par {" <expr> ("||" <expr>)$^{+}$ "}"
\end{grammar}
Parallel expression evaluation takes a sequence of routine calls and pauses the current routine's execution until all of the par'd routines return. For example:
\begin{lstlisting}
foo (a: Ref Int)  (b: Ref Int)
    par (bar a) || (baz b)
\end{lstlisting}
As per SSM Runtime semantics, \texttt{bar} is guaranteed to be executed first because the listed routine calls are evaluated in order.

The value of a par expression is a tuple containing the return values of the par'd routines.
\subsection{Waiting}
\begin{grammar}
<wait> ::= "wait {" <expr> ("||" <expr>)$^{+}$ "}"
\end{grammar}
Wait expressions take a sequence of expressions and pauses the routine's execution until any of the expressions is written to. Take for example the routine \texttt{wait_one_of_two}:
\begin{lstlisting}
wait_one_of_two (a: Sched Int)  (b: Sched Int)
    wait a || b
\end{lstlisting}
This routine will block until either \texttt{a} or \texttt{b} are written to, where \texttt{a} and  \texttt{b} are scheduled variables.

The value of a wait expression is unit. However, one can check which variable was written to using the \texttt{@} operator.
\subsection{Type-constrained expressions}
\begin{grammar}
<expr_ann> ::= <expr_ann> ":" <typ>
\end{grammar}

It is possible to type-annotate expressions for documentation purposes or to concretize type-generic expressions.

In the following code sample, we type-constrain a lambda expression such that it takes an integer, some other type, but returns an integer:
\begin{lstlisting}
((fun x y { x }): Int -> a -> Int)
\end{lstlisting}

\subsection{Instant and Delayed Reference Assignment}
\begin{grammar}
<assign> ::= <expr> "<- {" <expr> "}"
\alt "after" <expr> "," <expr> "<- {" <expr> "}"
\end{grammar}
References in sslang may be assigned to instantly or after some time period. In the following function definition, we showcase both forms of assignment.
\begin{lstlisting}
now_and_later (x: &Int) (y: &Bool)
    x <- let a = 2
         a + 5
    after 5 ms , y <- True
\end{lstlisting}
Note that similarly to let bindings, the \texttt{<-} assignment operator creates an indentation block, as seen in the instant assignment to \texttt{x}.

The value of assignment expressions is unit.
\subsection{Let Bindings}
\begin{grammar}
<let> ::= "let {" <def_let> ("||" <def_let>)$^{+}$ "}"; <expr>
\end{grammar}

In let blocks, patterns are bound to mutually-recursive expressions. Patterns have a wide range of use cases, from binding simple identifiers to deconstructing algebraic data types. In the let-block below, we demonstrate the use of patterns while binding three expressions:
\begin{lstlisting}
let x = 5                  // Simple binding
    (y, _) = foo           // Tuple structured binding
    f a b = let c = a + b  // Syntactic sugar for lambda definition
            c * 2
...
\end{lstlisting}
Note the careful use of indentation in the let block. In the absence of explicit braces, the pattern section creates an indentation block (hence \texttt{x}, \texttt{(y, _)}, and \texttt{f a b} appearing at the same level) and the expression value section creates an indentation block (hence the expressions \texttt{let c = ...} and \texttt{c * 2} appearing at the same level).

The value of a let-binding expression is the last expression in the sequence following the bindings.
\subsection{Sequences}
\begin{grammar}
<seq> ::= <expr_stm> ";" <expr>
\end{grammar}
Sequences chain expressions in block-expressions. The value of a sequence is the second expression in the sequence.
\begin{lstlisting}
(f 1; 5) // The value of the parenthesized expression is 5
\end{lstlisting}


\subsection{Pattern Matching}
\begin{grammar}
<match> ::=  "match" <expr_blk> "{" (  <pat> "=" "{" <expr> "}" )$^{+}$ "}"
\end{grammar}

Pattern matches are used to break down algebraic data types.
A pattern match consists of a ``scrutinee''---what is being matched on---and a number of ``arms.''
Each arm consists of a pattern that the scrutinee is matched against, and a ``consequence'' expression.
The value of the entire pattern match is the consequence of the first matching arm.
The types of each consequence must be the same in a pattern match, and the type of the scrutinee must match the type of the each pattern; the overall type of the match expression is that of its consequences.

In the example below, the \texttt{default0} function unwraps its option-typed argument, \texttt{o}.
In the case where \texttt{o} is \texttt{None}, \texttt{default0} defaults to returning a value of 0.

\begin{lstlisting}
type Option a
  Some a
  None

default0 (o: Option Int) -> Int =
  match o
    Some x = x
    None   = 0
\end{lstlisting}

\newpage

\hrule

\section{Wish List}

\subsection{Philosphy}

\textit{SE:} Make the IR, especially, as simple as possible (a la
GHC). Take a Reynolds-like approach where the larger language has more
features that boil down into simpler primitives.  E.g., typeclasses
can be expressed as dictionaries with function pointers and completely
evaporate.  I'm inspired by the make-it-explicit approach to garbage
collection operations in the Perceus tech report (Reinking et al.,
MSR-TR-2020-42).

\subsection{Syntax}

\textit{SE:} Minimal punctuation (e.g., print ``hello'' vs. print(``hello'')).
\textit{JH:} I would like to have punctuation for types/operations with special
built-in meaning, e.g., \verb|&| or \verb|*| for references. I'm not opposed to
using parens for function application, but it depends on whether we can have
partial application.

\textit{SE, JH:} Nested comments

\textit{JH:} Single-line comment syntax. I think this makes sense especially for
a whitespace-sensitive language.

\textit{SE, JH:} Indentation-sensitive nesting; keywords start blocks (Haskell-style).

\textit{SE, JH:} User-defined binary operators and precedence levels.

\textit{JH:} User-defined type-level operator definitions. e.g., \verb|$| for
avoiding parens in type application, \verb|&| for ref.

\textit{JH:} some kind of readable, forward-thinking syntax for annotations that
the compiler can take advantage of, e.g., must-inline, dont-inline,
stack-allocate, maybe-unused. Should be less hacky than Haskell's
\verb|{-# LANGUAGE ... #-}| and C/C++'s \verb|#pragma ..| and
\verb|__attribute__((..))|.

\textit{SE:} C\# has an elaborate attributes mechanism.
Most declarations allow attributes to be placed in square brackets before.  \url{https://docs.microsoft.com/en-us/dotnet/csharp/programming-guide/concepts/attributes/}

\begin{lstlisting}{language=[Sharp]C}
 
[Serializable]
public class SampleClass
{
    // Objects of this type can be serialized.
 
\end{lstlisting}

\subsubsection{Discussions}

\subsection{Types and expressions}

\textit{SE, JH:} Type expressions.

\textit{SE, JH:} Algebraic data types + pattern-matching.

Statically sized arrays passable through abstract array arguments, e.g., so you
can write a sort that's polymorphic in the size of the array; implies array size
is a runtime value, not (just) compile-time.

\textit{SE, JH:} Typeclasses at least for arithmetic operators

Pass-by-value and pass-by-reference values

\textit{JH:} References to arrays of T, arrays of arrays of T

\textit{JH, SE:} Array slices (``safe'' references to parts of arrays that play
well with parallelism)

\textit{SE, JH:} Type inference at least for local variables

\textit{JH:} Type annotations for function arguments

\textit{SE, JH:} Parametric polymorphism, likely eliminated by whole-program
monomorphisation (type variables)

\textit{SE, JH:} A notion of constants (assign once) versus variables (assign
multiple times)

\textit{JH:} a notion of ownership that users can opt into (which reduces to
reference-sharing otherwise).

\subsubsection{Discussions}

Partial application?  Or do we prohibit it?

\subsection{Control flow}

\textit{SE, JH:} Parallel function calls.

\textit{SE, JH:} Wait on variables

\textit{SE, JH:} Immediate and delayed assignment ``primitives''.

\textit{JH:} Some sort of exception handling, which supports missed deadlines.

\textit{SE, JH:} combinators to derive "wait until" and "wait both" and "wait either".

\subsection{Compilation and generated code}

\textit{SE, JH:} Compile-time allocation of all data (exception: activation records)

\textit{SE, JH:} Facility for efficient code generation with compile-time
constants (should this be syntactically different?).

\textit{JH:} provide some FFI that supports callback, to interface with both
blocking and non-blocking functions.

\textit{JH:} provide some kind of immediately effectful print statement.

\textit{SE:} No need for separate compilation, just do whole-program
compilation. This makes monomorphisation straightforward/simple.

\subsection{IR}

\begin{itemize}
\item Variables and literals
\item Lambda abstractions
\item Function application
\item Pattern matching
\end{itemize}

\section{Overview}

\begin{lstlisting}
main(led : Ref (Sched Bool) ) =
  loop
    after 50 ms, led <- True
    wait led
    after 50 ms, led <- False
    wait led
\end{lstlisting}

\begin{lstlisting}
toggle(led : Ref (Sched Bool)) =
  led <- not led

slow(led : Ref (Sched Bool)) =
  let e1 = Occur : Sched Event
  loop
    toggle led
    after 30 ms, e1 <- Occur
    wait e1

fast(led : Ref (Sched Bool)) =
  let e2 = Occur : Sched Event
  loop
    toggle led
    after 20 ms, e2 <- Occur
    wait e2
    
main(led : Ref (Sched Bool)) =
  pipe slow led
       fast led
\end{lstlisting}

\end{document}


loop
  f x
    y
  g x
h x

I0 loop P1
   { I1 f x
         y
   ;   g x
   } I0 ; h x


   (    loop
              wait    x    )

I0 ( B1 loop P2
         { I2 wait P3 { I3 x } I2 } B1 ) I0

N PendingBlockNL
I ImplicitBlock
B Block

-- add this aside from [ScannerContext] state
X EndingExplicitBlock


NOTE:
To find current block indentation, search up context stack until B of brace (no indentation) or I (take that indentation)


any time you find [ or ( delimiter, push B with that delimiter

if you find { in N, pop that state and then push B with {
    otherwise, just push B with {

if you find close delimiter ], ), or } I, emit }, and pop, then repeat without consuming close delimiter

if you find close delimiter ], ), or } in matching B, emit close delimiter
  if non-matching B, report syntax error about expecting matching delimiter

if you find close delimiter ], ), or } in N, report syntax error about expecting expression

if you across layout token (e.g., loop), scan to next token;
  if next token is {, emit {, push B
  if next token is anything else, make sure indentation of next token is greater than current block indentation, emit {, emit token, push I

if you across layoutNL token (e.g., if), push N on stack

newlines in B are ignored

newlines in N ...

newlines in I ...


---

take 2

states:

N PendingBlockNL of separator token
I ImplicitBlock of separator token and indentation level
B Block of delimiter token

definitions:

current block indentation := search up context stack until B of brace (no indentation) or I (take that indentation)

horizontal whitespace := spaces, tabs

whitespace := horizontal whitespace + newlines

rules:

next token is newline:
  scan to next non-whitespace token t but don't consume it

  current state is B:
    ignore -- explicit newlines don't mean anything in explicit blocks

  current state is I with separator s:
    if indentation of t > current block indentation:
      ignore -- this is a line continuation

    if indentation of t < current block indentation:
      emit '}'
      pop B from state stack and keep checking

    if indentation of t = current block indentation:
      emit s

  current state is N with separator s:
    assert indentation of t > current block indentation
    push I with separator s and indentation of t
    emit '{'

next token d is '[' or '(':
  push B with d

next token is '{':
  current state is N:
    pop state
    push B with '{'
    emit '{'

  otherwise:
    push B with '{'
    emit '{'

next token d is '}', ']', or ')':
  current state is I:
    pop state
    emit '}'
    unconsume d and repeat

  current state is B with matching delimiter:
    pop state
    emit d

  current state is B with non-matching delimiter:
    syntax error: unmatched delimiter d

  otherwise: (it was N)
    syntax error: expected expression

next token is layout token l (e.g., 'loop'):
  scan to next non-whitespace token t and consume it

  t is '{':
    push B with '{'
    emit l '{'

  t is anything else:
    assert indentation of t > current block indentation
    push I with separator of l and indentation of t
    emit l '{' t

next token is layoutNL token l (e.g., 'if'):
  push N with separator of l
  emit l


next token t is anything else:
  emit t

---

In Haskell, this is actually ok:

  do {(
return 0)}

loop
  a (wt
   x)

so we can't just check the top-most context's indentation level
